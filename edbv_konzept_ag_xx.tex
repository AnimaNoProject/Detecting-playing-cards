%%Berichtvorlage für EDBV WS 2015/2016

\documentclass[deutsch]{scrartcl}
\usepackage[ngerman]{babel}
\usepackage[utf8]{inputenc}
\usepackage{algorithmic}
\usepackage{algorithm}
\usepackage{graphicx}
\usepackage{amsmath,amssymb}
\usepackage{subcaption}
\captionsetup{compatibility=false}
\usepackage{multirow}
\usepackage{color}

\begin{document}

\title{Konzept: Projekttitel} %%Projekttitel hier eintragen

\subtitle{EDBV WS 2017/2018: AG\_A2} %%statt XX Arbeitsgruppenbezeichnung hier eintragen (zB.: A1)


%%Namen und Matrikelnummern der Gruppenmitglieder hier eintragen
\author{Jan Michael Lajarno (Matrikelnummer1)\\
Andreas Brunner (Matrikelnummer2)\\
Miran Jank (Matrikelnummer2)\\
Thorsten Korpitsch (01529243)\\
Aleksander Marinkovic (Matrikelnummer2)\\
}



%%------------------------------------------------------

\maketitle


%%------------------------------------------------------

(1-2 Seiten)
\section{Ziel}
Ziel des Projekts\\
\textit{Kommentar: erklärt euer Projekt in einem Satz}
\section{Eingabe}
Erwarteter Input (Datentyp, Eigenschaften, eventuell zusätzliche Parameter)\\
\textit{Kommentar: Welche Eingabe benötigt euer Programm, muss der User Parameter wählen, etc. ?}
\section{Ausgabe}
visuell:\\
textuell:\\
\textit{Kommentar: In welcher Form wird dem User der Output eures Programms präsentiert?}
\section{Voraussetzungen und Bedingungen}
Voraussetzungen für Eingabe definieren, anhand dieser werden Datensätze erstellt\\
\textit{Kommentar: Welche Eigenschaften muss der Input erfüllen, damit ihr damit arbeiten könnt (zB.: Kameraeinstellungen, Hintergrundeigenschaften, etc.)?}
\section{Methodik}
Methodik- Pipeline
\begin{enumerate}
	\item Methode 1
	\item Methode 2
	\item ...
\end{enumerate}
\textit{Kommentar: Die folgenden Fragen sollten hier bedacht und beantwortet werden: Welche Arbeitsschritte sind notwendig um für den gegebenen Input den entsprechenden Output zu berechnen? Wozu sind die jeweiligen Methoden notwendig – d.h. welche konkrete Methoden wird für diese Arbeitsschritte verwendet?}
\section{Evaluierung}
\begin{itemize}
	\item Evaluierungsfrage 1
	\item Evaluierungsfrage 2
	\item ...
\end{itemize}
\textit{Kommentar: Eine qualitative Evaluierung basiert auf der subjektiven Wahrnehmung einer Person (Ist ein Ergebnis gut oder schlecht?). Ihr sollte hier aber vor allem auch eine quantitative Evaluierung durchführen, d.h. eine objektive Evaluierung durch Vergleich eurer (Zwischen-)Ergebnisse mit ground truth (ein klassisches Beispiel: Für wie viele der Test-Datensätze wurde für Aufgabe xy ein korrektes Ergebnis erzielt?).}
\section{Datenbeispiel}
\begin{figure}[h!]
 \centering
 \includegraphics[width=0.4\textwidth]{img.jpg}
 \caption{puppy}
 \label{fig:img}
\end{figure}
\section{Zeitplan}
\begin{table}[h!]
	\centering
		\begin{tabular}{|c|c|c|}
		\hline
		Meilenstein & abgeschlossen am & Arbeitsaufwand in h\\
		\hline
		...&... &...\\
		\hline
		\end{tabular}
\end{table}
\textit{Kommentar: Definiert euch „Meilensteine“. Die vorgegebenen Termine (zB. Zwischenpräsentation) sind hier nicht von Interesse, stellt euch eher die Frage: Wann rechnet ihr mit einem fertigen Prototyp (mit Hilfe von Matlab-Toolboxes)? Wann soll danach ein gewisser Arbeitsschritt (entsprechend eurer Methodik-Pipeline) fertig implementiert sein? Plant auch Zeit für zB. Tests, Evaluierung etc. ein.
Gebt weiters pro Arbeitsschritt an, wieviel Arbeitsaufwand (Stunden) eurer Meinung nach zur Umsetzung notwendig sind. Bedenkt, dass es sich bei EDBV um eine Übung im Ausmaß von 3.0 ECTs handelt. Für 1.0 ECTs rechnet man mit 25h Arbeitsaufwand pro Semester. Auf Teil1 (die Gruppenphase von EDBV) entfallen 2.4 dieser 3.0 ECTs und somit 60h Arbeit pro Gruppenmitglied. Wir rechnen daher für Teil 1 mit 300h Arbeit pro Gruppe.
}\\
\\
\textit{Kommentar: Gebt eine relevante Literaturquelle (Bücher bzw. Kapitel, Konferenz- oder Journal-Papers) pro Gruppenmitglied (im kompilierten Bibtex-Format - Beispiele für Referenzen im Bibtex-Format: http://verbosus.com/bibtex-style-examples.html?lang=de). Diese Quellen sollten für euch bei zB. der Implementierung einer Methode, der Wahl von Parametern, etc. helfen. Können aber auch ein ähnliches Problem behandeln und motivieren, warum ihr euch für gewisse Methodik entschieden habt.}
%%------------------------------------------------------
\bibliographystyle{plain}
\nocite{*}
\bibliography{edbv_lit}
%%Bei verwendung von Latex schreibt ihr eure Referenzen in ein eigenes bib-File (siehe hier Beispiel in edbv_lit.bib). Weitere Information zum Einbinden von BibTex gibt es hier: http://www.bibtex.org/Using/de/
%%------------------------------------------------------

\end{document}
