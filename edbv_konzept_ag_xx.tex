\documentclass[deutsch]{scrartcl}
\usepackage[ngerman]{babel}
\usepackage[utf8]{inputenc}
\usepackage{algorithmic}
\usepackage{algorithm}
\usepackage{graphicx}
\usepackage{amsmath,amssymb}
\usepackage{subcaption}
\captionsetup{compatibility=false}
\usepackage{multirow}
\usepackage{color}

\begin{document}

\title{Konzept: Projekttitel} %%Projekttitel hier eintragen

\subtitle{EDBV WS 2017/2018: AG\_A2} %%statt XX Arbeitsgruppenbezeichnung hier eintragen (zB.: A1)


%%Namen und Matrikelnummern der Gruppenmitglieder hier eintragen
\author{Jan Michael Lajarno (01425799)\\
Andreas Brunner (01429369)\\
Miran Jank (01526438)\\
Thorsten Korpitsch (01529243)\\
Aleksandar Marinkovic (01634028)\\
}



%%------------------------------------------------------

\maketitle


%%------------------------------------------------------

\section{Ziel}
Ziel des Projekts\\
\textit{Der Gewinner einer Runde wird erkannt(der Stich) und die Punkte werden dem Gewinner zugerechnet. Am Ende wird der Gewinner gekührt.}
\section{Eingabe}
Erwarteter Input (.PNG/.JPG/.JPEG, Farbbild der Karten)\\
\textit{Der Benutzer muss nur ein Farbbild der Karten (aufgereiht) bereitstellen.}
\section{Ausgabe}
textuell:\\
\textit{In der Konsole wird ein Zwischenstand nach jeder Runde ausgegeben, am Ende wird der Gewinner ausgegeben und der Endstand.}
\section{Voraussetzungen und Bedingungen}
Voraussetzungen für Eingabe definieren, anhand dieser werden Datensätze erstellt\\
\textit{Ein Farbbild der Karten. Der Hintergrund sollte möglichst Einfarbig sein (nicht weiß, texturarm). Die Kamera soll sich in einem Winkel von 45 bis 135 Grad befinden. Die Karten müssen mit einem dünnen schwarzen Rand preperiert sein.}
\section{Methodik}
Methodik- Pipeline
\begin{enumerate}
	\item Geometrische Transformation
	\item Kanten-Filter
	\item Template-Matching
\end{enumerate}
\section{Evaluierung}
\begin{itemize}
	\item Um das Eingabebild vorzubereiten wird aus dem, bis zu 45 Grad schrägen Bild, ein Bild aus der Vogelperspektive (90 Grad) transformiert
	\item Um die einzelnen Spielkarten zu trennen wird eine Kantendetektion durchgeführt
	\item Um die Spielkarte zu identifizieren wird ein Pattern-Matching durchgeführt
\end{itemize}
\section{Datenbeispiel}
\begin{figure}[h!]
 \centering
 \includegraphics[width=0.4\textwidth]{datenbeispiel.jpg}
 \caption{Datenbeispiel: Kreuz-Bube sticht Herz-König}
 \label{fig:img}
\end{figure}
\newpage
\section{Zeitplan}
\begin{table}[h!]
	\centering
		\begin{tabular}{|c|c|c|}
		\hline
		Meilenstein & abgeschlossen am & Arbeitsaufwand in h\\
		\hline
		Vorarbeit Projekt & 20.10.2017 & 20\\
		Prototyp erstellen& ... & 40\\
		Geometrische Transformation& ... & 105 \\
		GT-Test & ... & 10 \\
		Kantendetektion& ... & 67 \\
		KD-Test & ... & 5 \\
		Pattern-Matching & ... & 43\\
		PM-Test & ... & 5 \\
		Unit-Test & ... & 5 \\
		\hline
		\end{tabular}
\end{table}
\nocite{*}
\bibliographystyle{plain}
\bibliography{C:/Users/Thorsten/Documents/GitHub/EDBV/edbv_lit}{}
\end{document}
